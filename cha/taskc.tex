%!TEX root = ../main.tex
\section{Task C}
For realising the aforementioned business model we create the following dimensions:
member, product, date and time.
The sale information from the database will be what binds the data together in the sales fact table. 

To represent our data warehouse in a relational database, we choose to use a star schema.
We do that since it will be easier for us later to create our cube in Mondrian.

Below the created dimensions and star schema are shown.


%member - member.id
%product - product.id, product.price
%sale - sale.member_id, sale.product_id, sale.timestamp, sale.price 
%date - year, month, day 
%time - hour, minute, seconds

\subsection{Dimensions}

\begin{figure}[H]
    \centering
    \begin{minipage}[b]{0.4\textwidth}
        \begin{tikzpicture}
            \umlclass{T}{}{}
            \umlclass[y=-3]{Year}{}{}
            \umlclass[y=-6]{Month}{Season \\ Month in year \\ Days in month}{}
            \umlclass[y=-10]{Day}{Day in month \\ Weekday \\ Holiday}{}
            
            \umlassoc[mult1=1..1, mult2=1..*]{T}{Year}
            \umlassoc[mult1=1..1, mult2=12]{Year}{Month}
            \umlassoc[mult1=1..1, mult2=28..31]{Month}{Day}
        \end{tikzpicture}
        \caption{Date dimension.}
        \label{fig:date_dim}
    \end{minipage}
    \begin{minipage}[b]{0.4\textwidth}
        \begin{tikzpicture}
            \umlclass{T}{}{}
            \umlclass[y=-3]{Hour}{}{}
            \umlclass[y=-6]{Minut}{}{}
            \umlclass[y=-9]{Second}{}{}
            
            \umlassoc[mult1=1..1, mult2=1..*]{T}{Hour}
            \umlassoc[mult1=1..1, mult2=60]{Hour}{Minut}
            \umlassoc[mult1=1..1, mult2=60]{Minut}{Second}
        \end{tikzpicture}
        \caption{Time dimension.}
        \label{fig:time_dim}
    \end{minipage}
\end{figure}

\begin{figure}[H]
    \centering
    \begin{minipage}[b]{0.4\textwidth}
        \begin{tikzpicture}
            \umlclass{T}{}{}
            \umlclass[y=-3]{Price}{}{}
            \umlclass[y=-6]{Product}{Name}{}
            
            \umlassoc[mult1=1..1, mult2=1..*]{T}{Price}
            \umlassoc[mult1=1..1, mult2=1..*]{Price}{Product}
        \end{tikzpicture}
        \caption{Product dimension.}
        \label{fig:product_dim}
    \end{minipage}
    \begin{minipage}[b]{0.4\textwidth}
        \begin{tikzpicture}
            \umlclass{T}{}{}
            \umlclass[y=-3]{Year}{}{}
            \umlclass[y=-6]{Member}{Balance \\ Active}{}
            
            \umlassoc[mult1=1..1, mult2=1..*]{T}{Year} %Tjek
            \umlassoc[mult1=1..1, mult2=0..*]{Year}{Member}
        \end{tikzpicture}
        \caption{Member dimension.}
        \label{fig:member_dim}
    \end{minipage}
\end{figure}


\subsection{Star schema}
Below the star schema can be seen.
The data content of the schemes is for display purpose. 

\begin{figure}[H]
    \centering
    \begin{tabular}{|l|l|l|l|}
        \hline
        \underline{memberid}    & year  \\ \hline
        1061                    & 1996  \\ \hline
        1060                    & 1996  \\ \hline
    \end{tabular}
    \caption{Member dimension table.}
    \label{fig:member_scheme}

\vspace{0.5cm}

    \begin{tabular}{|l|l|l|l|}
        \hline
        \underline{dateid}      & year      & month & day   \\ \hline
        1                       & 2007      & 1     & 5     \\ \hline
        2                       & 2005      & 5     & 7     \\ \hline
    \end{tabular}
    \caption{Date dimension table.}
    \label{fig:day_scheme}

\vspace{0.5cm}

    \begin{tabular}{|l|l|l|l|}
        \hline
        \underline{timeid}      & hour      & minute    & second    \\ \hline
        5                       & 5         & 10        & 55        \\ \hline
        9                       & 9         & 45        & 8         \\ \hline
    \end{tabular}
    \caption{Time dimension table.}
    \label{fig:time_scheme}

\vspace{0.5cm}

    \begin{tabular}{|l|l|l|l|l|}
        \hline
        \underline{productid}   & name                  & price     \\ \hline
        5                       & $^1/_2$L Kærnemælk    & 225       \\ \hline
        40                      & Frisk luft v. BSS     & 100       \\ \hline
    \end{tabular}
    \caption{Product dimension table.}
    \label{fig:product_scheme}

\vspace{0.5cm}

    \begin{tabular}{|l|l|l|l|l|l|}
        \hline
        \underline{memberid}    & \underline{dayid}     & \underline{timeid}  & \underline{productid}  & Price    & sale  \\ \hline
        1061                    & 1                     & 5                   & 5                      & 500      & 20    \\ \hline
        1060                    & 2                     & 9                   & 40                     & 800      & 30    \\ \hline
    \end{tabular}
    \caption{Sale fact table.}
    \label{fig:sale_scheme}
\end{figure}